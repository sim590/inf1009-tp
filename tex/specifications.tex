%-------------------------------------------------------------
% Fichier: specifications.tex    Auteur(s): Simon DÉSAULNIERS
% Date: 2013-04-24
%-------------------------------------------------------------
% Document de spécification du code pour le projet de
% session du cours INF1009 à l'Université du Québec à
% Trois-Rivières
%-------------------------------------------------------------
\documentclass[11pt,french]{article}

\usepackage[frenchb]{babel}
\usepackage[utf8]{inputenc}
\usepackage[T1]{fontenc}

\usepackage{graphicx}

\usepackage{hyperref}
\hypersetup{hidelinks}

% Style des pages
\usepackage{fancyhdr}
\pagestyle{fancy}
\lhead{Travail de Session}

% Caractères mathématiques
\usepackage{amssymb}

\usepackage{color}

% Affichage de code source
\usepackage{listings}
\definecolor{dkgreen}{rgb}{0,0.6,0}
\definecolor{gray}{rgb}{0.5,0.5,0.5}
\definecolor{mauve}{rgb}{0.58,0,0.82}

\lstset{ %
backgroundcolor=\color{white},   % choose the background color; you must add \usepackage{color} or \usepackage{xcolor}
basicstyle=\footnotesize,        % the size of the fonts that are used for the code
breakatwhitespace=false,         % sets if automatic breaks should only happen at whitespace
breaklines=true,                 % sets automatic line breaking
captionpos=b,                    % sets the caption-position to bottom
commentstyle=\color{green},    % comment style
deletekeywords={...},            % if you want to delete keywords from the given language
escapeinside={\%*}{*)},          % if you want to add LaTeX within your code
extendedchars=true,              % lets you use non-ASCII characters; for 8-bits encodings only, does not work with UTF-8
frame=single,                    % adds a frame around the code
keywordstyle=\color{blue},       % keyword style
morekeywords={*,...},            % if you want to add more keywords to the set
numbers=left,                    % where to put the line-numbers; possible values are (none, left, right)
numbersep=5pt,                   % how far the line-numbers are from the code
numberstyle=\tiny\color{gray}, % the style that is used for the line-numbers
rulecolor=\color{black},         % if not set, the frame-color may be changed on line-breaks within not-black text (e.g. comments (green here))
showspaces=false,                % show spaces everywhere adding particular underscores; it overrides 'showstringspaces'
showstringspaces=false,          % underline spaces within strings only
showtabs=false,                  % show tabs within strings adding particular underscores
stepnumber=2,                    % the step between two line-numbers. If it's 1, each line will be numbered
stringstyle=\color{mauve},     % string literal style
tabsize=2,                       % sets default tabsize to 2 spaces
xleftmargin=\parindent,
title=\lstname                   % show the filename of files included with \lstinputlisting; also try caption instead of title
}


% Style des en-têtes de section
\usepackage{sectsty}
%\sectionfont{\sectionrule{0ex}{0pt}{-1ex}{0.5pt}}

\newcommand{\HRule}{\rule{\linewidth}{0.5mm}}

\begin{document}
    % PAGE TITRE
    \begin{titlepage}
        \begin{center}
    
            \includegraphics[height=3cm]{./aux/network.png}
            \\[3cm]
            
            \textsc{\LARGE Réseaux d'ordinateurs I}
            \\[0.2cm]
            \textsc{\Large INF1009}
            \\[2cm]
            \HRule \\[0.5cm]
            {\huge \bfseries Travail de Session}
            \HRule \\[2cm]
            Par\\
            Simon Désaulniers

            \vfill
            2013-04-24\\
            Université du Québec à Trois-Rivières
            \thispagestyle{empty}
        \end{center}
    \end{titlepage}
    \newpage

    % TABLE DES MATIÈRES
    \pagenumbering{roman}
    \setcounter{page}{1}
    \tableofcontents
    \newpage

    % CORPS DU RAPPORT
    \pagenumbering{arabic}
    \setcounter{page}{1}
    
    \section*{Introduction} % (fold)
    \label{sec:intro}
        Le document présent est destiné à la meilleure compréhension de la conception
        du programme réalisé pour le cours \emph{INF1009}. Les différents fichiers sources
        \footnote{Le code source est disponible à l'adresse {\color{blue}\href{https://github.com/sim590/inf1009-tp}{https://github.com/sim590/inf1009-tp}}}
        seront détaillés afin d'expliquer les choix qui ont été faits lors de l'écriture
        de ceux-ci.\\

        Le projet est réalisé suivant les spécifications de l'éconcé en annexe et donc demeure
        fonctionnel dans un environnement utilisant la norme \emph{POSIX}\footnote{Comme les
        différentes distributions \emph{GNU/Linux} (Debian, Ubuntu, Linux Mint, etc.)}.
    % section intro (end)
    
    \section{Makefile} % (fold)
    \label{sec:makefile}
        Le fichier nommé \emph{Makefile} permet de compiler\footnote{La compilation utilise le programme
        gcc (GNU C/C++ Compiler).} les différents fichiers sources
        à l'aide d'une seule commande entrée au terminal: {\bf make all}. Lorsque cette commande
        est envoyée au terminal et que répertoire de travail est le même que le fichier \emph{Makefile},
        la règle \emph{all} est appellée.
        \lstset{language=make}
        \begin{lstlisting}
all: main transport network
        \end{lstlisting}
        Celle-ci appelle ses dépendences et ainsi compile tous les 
        fichiers sources.
        %\lstinputlisting{../Makefile}

        Une fois les fichiers compilés et liés dans 3 objets binaires
        {\bf inf1009-tp}, {\bf transport-entity} et {\bf network-entity},
        il est possible de lancer le programme en appellant\footnote{Il est recommandé de
        démarrer le programme à l'aide d'un terminal afin d'avoir différentes informations sur
        la sortie standard.} le programme \emph{inf1009-tp}.
        Bien-sûr, le programme nécessite le fichier \emph{S\_LEC} afin d'obtenir des informations et
        ainsi produire les résultats attendus dans les autres fichiers \emph{S\_ECR}, \emph{L\_LEC}
        et \emph{L\_ECR}.\\

        Le script permet aussi de nettoyer le répertoire des fichiers sources et des fichiers binaires
        en utilisant respectivement la commande {\bf make clean} et {\bf make cleanbin}
    % section makefile (end)
    
    \section{inf1009-tp} % (fold)
    \label{sec:inf1009-tp}
        \lstset{language=c}
        Ce programme est comilé à l'aide du fichier \emph{main.c}. C'est le programme qui appelle
        les deux processus ET\footnote{Entité Transport} et ER\footnote{Entité Réseau}. Il nécessite
        que les deux programmes soient dans le même répertoire.
        
        \subsection{Fichiers d'entêtes} % (fold)
        \label{sub:fich-entete}
            
            \subsubsection{main.h} % (fold)
            \label{ssub:main.h}
                Le fichier \emph{main.h} contient les appels aux librairies nécessaires ainsi que la déclaration
                d'une fonction à laquelle on fait appel dans le fichier \emph{main.c}.
            % subsubsection main.h (end)
        % subsection fich-entete (end)
        \subsection{Fichiers compilés} % (fold)
        \label{sub:fich-compiles-inf1009-tp}
        
        % subsection fich-compiles (end)
    % section inf1009-tp (end)

end{document}
